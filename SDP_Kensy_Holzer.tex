
\documentclass[12pt,a4paper]{scrartcl}

\usepackage[a4paper, left=2cm, right=1cm, bottom=1cm, top=1cm, includeheadfoot]{geometry}
\usepackage[ngerman]{babel}
\usepackage[utf8]{inputenc} % comment this if you uncomment utf8x
%\usepackage[utf8x]{inputenc} % uncomment this if there are problems with 'ä', 'ü', 'ö'
\usepackage{ucs}
\usepackage[usenames,dvipsnames]{xcolor}
\usepackage[fleqn]{amsmath}
\usepackage{amsfonts}
\usepackage{amssymb}
\usepackage{color}
\usepackage{listings}
\usepackage{hyperref}
\usepackage{amsfonts}
\usepackage{pdfpages}
\usepackage{listings}
\usepackage{scrpage2}
\usepackage{graphicx}


\definecolor{mygray}{rgb}{0.9,0.9,0.9}
\lstset{language=[Visual]Basic, morekeywords={param, local}}


\lstset{
   literate={ö}{{\"o}}1
           {ä}{{\"a}}1
           {ü}{{\"u}}1
           {ß}{{\ss}}1
           {é}{{\'e}}1,
   inputencoding=ansinew,
   extendedchars=true,
   basicstyle=\scriptsize\ttfamily,
   numberstyle=\scriptsize,
   breaklines=true,
   tabsize=2,
   numbersep=5pt
}
\lstdefinestyle{customcpp}{
   language=C++,
   backgroundcolor=\color{mygray},
   numbers=left,
   keywordstyle=\color{blue}\bfseries,
   stringstyle=\color{BrickRed}\ttfamily,
   commentstyle=\color{OliveGreen}\ttfamily,
   showspaces=false,
   showstringspaces=false,
   showtabs=false
}
\lstdefinestyle{customoutput}{
   backgroundcolor=\color{mygray},
   numbers=none,
   showspaces=false,
   showtabs=false
}

\newcommand{\sourceCode}[1]{\lstinputlisting[style=customcpp]{#1}}
\newcommand\tab[1][1cm]{\hspace*{#1}}
%****************** self defined commands ***************************
%include a .cpp source file
%usage: \cppfile{../path_to_your_file/filename}
\newcommand{\cppfile}[1]{
\begin{itemize}
\item[]\lstinputlisting[caption="#1.cpp",label="#1.cpp", style=customcpp]{#1.cpp}
\end{itemize}
}

%include a .c source file
%usage: \cfile{../path_to_your_file/filename}
\newcommand{\cfile}[1]{
\begin{itemize}
\item[]\lstinputlisting[caption="#1.c",label="#1.c", style=customcpp]{#1.c}
\end{itemize}
}

%include a code snippet
%usage: \hfile{../path_to_your_file/filename}{startline}{endline}
\newcommand{\csnippet}[3]{
\begin{itemize}
\item[]\lstinputlisting[caption=#1.c, style=customcpp, firstline=#2,lastline=#3]{#1.c}
\end{itemize}
}

%include a .h source file
%usage: \hfile{../path_to_your_file/filename}
\newcommand{\hfile}[1]{
\begin{itemize}
\item[]\lstinputlisting[caption="#1".h,label="#1.h", style=customcpp]{#1.h}
\end{itemize}
}

%include a module (.cpp and .h)
%usage: \hfile{../path_to_your_file/filename}
\newcommand{\cppmodul}[1]{
\begin{itemize}
\item[]\lstinputlisting[caption=#1.h,label=#1.h, style=customcpp]{#1.h}
\item[]\lstinputlisting[caption=#1.cpp,label=#1.cpp, style=customcpp]{#1.cpp}
\end{itemize}
}

%include a module (.c and .h)
%usage: \hfile{../path_to_your_file/filename}
\newcommand{\cmodul}[1]{
\begin{itemize}
\item[]\lstinputlisting[caption=#1.h,label=#1.h, style=customcpp]{#1.h}
\item[]\lstinputlisting[caption=#1.c,label=#1.c, style=customcpp]{#1.c}
\end{itemize}
}

%include a code snippet
%usage: \hfile{../path_to_your_file/filename}{startline}{endline}
\newcommand{\snippet}[3]{
\begin{itemize}
\item[]\lstinputlisting[caption=#1.cpp,label=#1.cpp, style=customcpp, firstline=#2,lastline=#3]{#1.cpp}
\end{itemize}
}

%include a test-output file 
%usage: \testoutput{../path_to_your_file/filename}
\newcommand{\testoutput}[1]{
\begin{itemize}
\item[]\lstinputlisting[caption=#1]{#1.txt}
\end{itemize}
}

%include a .s source file
%usage: \assemblerfile{../path_to_your_file/filename}

\newcommand{\assemblerfile}[1]{
\begin{itemize}
\item[]\lstinputlisting[caption=#1.s ,label=#1.s , style=customassembler]{#1.s}
\end{itemize}
}

%include a code snippet
%usage: \hfile{../path_to_your_file/filename}{startline}{endline}
\newcommand{\assemblersnippet}[3]{
\begin{itemize}
\item[]\lstinputlisting[caption=#1.s, style=customassembler, firstline=#2,lastline=#3]{#1.s}
\end{itemize}
}


 %beinhaltet alle benötigten Packages etc.
\begin{document}

\title{SDP - Uebung 1} % Übungsname und Nummer angeben
\subtitle{Wintersemester 2019/20} % Semester angeben oder auskommentieren, falls nicht erwünscht
\author{Adam Kensy - S1810306018\\
		Philipp Holzer - S1810306028} % Autorenname
\date{\today}
%date{} % Das heutige Datum automatisch einfügen


\begin{textblock*}{40mm}(45mm,67mm)
Adam Kensy	
\end{textblock*}
\begin{textblock*}{40mm}(45mm,81mm)
Philipp Holzer
\end{textblock*}
\begin{textblock*}{40mm}(55mm,96mm)		
2
\end{textblock*}
\begin{textblock*}{40mm}(78mm,110mm)
14
\end{textblock*}
\begin{textblock*}{40mm}(170mm,110mm)
16
\end{textblock*}
\includepdf[pages=-]{angabe}


\maketitle % Titelseite erstellen


\tableofcontents % Inhaltsverzeichnis erstellen

\newpage


% Kure Befehlsreferenz
%\section{} erstellt eine Überschrift
%während
%\subsection{} eine Unterüberschrift erstellt
% Eine neue Seite wird mit \newpage erstellt

\section{Organisatorisches}

\subsection{Team}
\begin{itemize}
	\item Philipp Holzer, Matr.-Nr.: 1810306028
	\item Adam Kensy, Matr.-Nr.: 1810306018
\end{itemize}

\subsection{Aufteilung und Verantwortlichkeitsbereiche}
\begin{itemize}
	\item Philipp Holzer
		\begin{itemize}
		\item Planung
		\item Klassendiagramm
		\item Implementierung und Testen der Klassen
			\begin{itemize}
			\item Logbook
			\item Vehicle
			\end{itemize}
		\item Dokumentation
		\end{itemize}
	\item Adam Kensy
		\begin{itemize}
		\item Planung
		\item Klassendiagramm
		\item Implementierung und Testen der Klassen
			\begin{itemize}
			\item Carpool
			\item Vehicle
			\item Car, Truck, Motorcycle
			\end{itemize}
		\item Dokumentation
		\end{itemize}
\end{itemize}

\subsection{Aufwand}
\begin{itemize}
	\item Philipp Holzer		\tab geschätzt: 7	\tab tatsächlich: 8
	\item Adam Kensy		\tab  geschätzt:	7	\tab  tatsächlich: 8
\end{itemize}

% Für eine numerierte Aufzählung verwendet man 

%\begin{enumerate}
%	\item 
%\end{enumerate}

%%%%%%%%%%%%%%%%%%%%%%%%%%%%%%%
\newpage
\section{Anforderungsdefinition (Systemspezifikation)}

Gesucht ist ein Fuhrparkverwaltungssystem, welche verschieden Fahrzeuge und deren Fahrtenbücher verwaltet.\\
Dabei gibt es drei unterschiedliche Fahrzeugarten - PKW, LKW und Motorrad - wo jedes jeweils ein Kennzeichen, eine Marke und ein Fahrtenbuch hat.\\
Das Fahrtenbuch wird chronologisch abgespeichert - man kann auch Einträge entfernen. 


%%%%%%%%%%%%%%%%%%%%%%%%%%%%%%%
\section{Systementwurf}

\subsection{Klassendiagramm}
 \includegraphics[scale = 0.55]{UML-Diagram.pdf}

\subsection{Klassendiagramm durch Visual Studio}
\includegraphics[scale=0.65]{./CarPool/ClassDiagram.png}

%%%%%%%%%%%%%%%%%%%%%%%%%%%%%%%
\subsection{Komponentenübersicht}

\begin{itemize}

\item \textbf{Klasse \string"Object"} \\
Basis aller Klassen

\item \textbf{Klasse "Carpool"}
\\ Verwaltet alle Fahrzeuge
\\ Besitzt eine Ausgabefunktion um alle enthaltenen Fahrzeuge auszugeben

\item \textbf{Klasse "Logbook"}
\\ Das Fahrtenbuch der Fahrzeuge
\\ Vor der Ausgabe wird das Fahrtenbuch immer nach Datum sortiert

\item \textbf{Klasse "Vehicle"}
\\ Stellt die Fahrzeuge dar, dazu gehören: PKW, LKW, Motorräder

\item \textbf{Klasse "Car", "Truck", "Motorcycle"}
\\ Konkrete Objekte für die Fahrzeuge
\\ Besitzen nur eine Ausgabefunktion, wobei der Ausgabeoperator überschrieben ist

\end{itemize}

\subsection{Designentscheidungen}
\begin{itemize}
\item Es wurde keine EBNF erstellt da es international anwendbar sein soll. 
\item Der Fuhrpark redet mit uns über die Konsole wenn etwas schiefläuft -> wenn kein Fahrzeug hinzugefügt oder entfernt werden konnte bekommen wird eine Fehlermeldung aber ansonsten bleibt der Fuhrpark ruhig. 
\item Wir haben uns entschieden, dass das Einfügen und Entfernen über Pointer läuft - wir dachten uns, es macht so am meisten Sinn denn man parkt das Fahrzeug irgendwo und teilt dies dann der Verwaltung mit. 
\item Es wurde eine Liste genommen, da dadurch immer wieder ein Fahrzeug hinzugefügt werden kann, somit kann es ständig wachsen. Außerdem lässt sich in einer Liste ein Fahrzeug auch schneller finden. 
\item Dadurch, dass wir mit Pointer arbeiten haben wir beim Carpool die Rule-Of-Three angewandt. 
\item Wir benutzen die tm-Struktur für Datums-Angaben und erwarten uns eine sinnvolle Eingabe vom Anwender. 
\item Es wird sortiert ins Fahrtenbuch eingetragen, dies macht am meisten Sinn, da man dies direkt nach einer Fahrt macht. 
\end{itemize}

%%%%%%%%%%%%%%%%%%%%%%%%%%%%%%%
%\section{Dateibeschreibung}

%Hier werden die verwendeten Dateien beschrieben, sofern Informationen aus externen
%Dateien gelesen bzw. in diese geschrieben werden. Eine solche Beschreibung umfasst...

%%%%%%%%%%%%%%%%%%%%%%%%%%%%%%%


%%%%%%%%%%%%%%%%%%%%%%%%%%%%%%%
\newpage
\section{Komponentenentwurf}

\subsection{Klasse Object}
Diese Klasse stellt die Basis aller Klassen dar.

\subsection{Klasse CarPool}
Diese Klasse verwaltet alle Fahrzeuge im Fuhrpark. Die Pointer auf alle Fahrzeuge werden in einer Liste gespeichert. \\
Hier wurde die Rule-Of-Three angewandt, damit man einen Fuhrpark kopieren beziehungsweise einem anderen Objekt zuweisen kann.\\
Mit AddVehicle() können Fahrzeuge hinzugefügt werden. Dieser Funktion muss ein Pointer auf ein PKW, LKW oder Motorrad übergeben werden. \\
Mit RemoveVehicle() können Fahrzeuge entfernt werden und dieser Funktion muss ein Kennzeichen übergeben werden. \\
Mit SearchByLicense() kann nach Fahrzeugen gesucht werden. Dieser Funktion wird ebenfalls ein Kennzeichen übergeben. Wurde das Kennzeichen im Fuhrpark gefunden wird true zurückgegeben ansonsten false, was und auch in der Konsole mitgeteilt wird. \\
Mit der PrintVehicles() Funktion wird der Fuhrpark komplett ausgegeben. Hierbei wird die Print Funktion jedes Fahrzeuges aufgerufen worin auch das LogBook ausgegeben wird.

\subsection{Klasse Vehicle}
Diese Klasse, von der die einzelnen Fahrzeuge abgeleitet werden, besitzt alle Basis-Funktionen für ein Fahrzeug.\\
Mit den beiden Get-Funktionen (GetBrand() und GetLicense()) erfährt man die Marke und das Kennzeichen des Autos. Die Print()-Funktion ist eine virtuelle Methode welche dann in den einzelnen Fahrzeugen aufgerufen wird. \\
Die virtuelle Clone()-Funktion wurde zwecks des Copy-Constructur der Klasse "CarPool" hinzugefügt wo ein Pointer der neu erstellten Kopie eines Fahrzeug-Objektes zurückgegeben wird. 

\subsection{Klasse Car, Truck und Motorcycle}
Die von der Fahrzeug abgeleiteten Klassen sehen beinahe identisch aus. Alle 3 haben ihren Konstruktur wo man ein Kennzeichen und eine Marke übergibt.\\
Die überschriebene Print()-Funktion wird von deren Basisklasse zuerst aufgerufen. Dies trifft auch auf die Clone()-Funktion zu. 

\subsection{Klasse LogBook}
Diese Klasse stellt das Fahrtenbuch eines Fahrzeuges dar. Mit NewEntry() kann ein Eintrag getätigt werden, wobei sortiert eingefügt wird und mit RemoveEntry() kann man einen Eintrag entfernen. Beide verlangen als Parameter ein Datum des Datentyp "tm" (aus der library <ctime>). \\
Die PrintLogs()-Funktion wird in den einzelnen Print-Funktionen der abgeleiteten Fahrzeugen aufgerufen, sodass das eigene Fahrtenbuch formatiert ausgegeben wird. \\
Mit Clear() kann ein ganzes Fahrtenbuch gelöscht werden. Die GetKMSum() Funktion berechnet die gesamt gefahrenen Kilometer eines Fahrzeuges.


%%%%%%%%%%%%%%%%%%%%%%%%%%%%%%%
\newpage
\section{Testprotokollierung}

%Im Testprotokoll werden die Testdaten und die Testergebnisse für alle Testfälle beschrieben. Weiters muss die Testumgebung angeführt sein (welches %Testframework
%wurde verwendet, mit welchen Komponentenversionen wurde gestestet, welche Stubs
%wurden verwendet, etc). Wenn die Komponenten und Subsysteme getrennt getestet
%wurden, ist die Testprotokollierung für die Komponenten getrennt anzugeben. Weiters sind identifizierte Schwachstellen und Probleme festzuhalten.

\subsection{Testumgebung}

Microsoft Visual Studio Enterprise 2019\\
Version 16.3.5\\
Microsoft Visual C++ 2019\\
\\
Windows 10, 64Bit, Build 18362\\
\\
Testdriver: main.cpp

\subsection{Testausgabe}
\lstinputlisting{test_output.txt}

%%%%%%%%%%%%%%%%%%%%%%%%%%%%%%%
%\section{Tabellen und Diagramme}

%Ergänzende und unterstützende Tabellen und Diagramme können am Ende der Systemdokumentation angefügt werden.

% Zur Info: Tabellen am besten in Excel erstellen und dann als Bild hier einfügen - Tabellen können in Latex sehr unangenehm werden!!
% Befehl um Bilder einzufügen:
%\includegraphics{Relativer/Pfad/zum/Bild.Endung}

%%%%%%%%%%%%%%%%%%%%%%%%%%%%%%%
%\section{Software-Qualitätsmetriken}

%Metriken dienen dazu, die Qualität von Software zu messen. Für C++ gibt es hier
%etwa das frei verfügbare Programm CCCC, welches für Linux und Windows unter
%der Adresse http://sourceforge.net/projects/cccc verfügbar ist. Für einen gegebenen
%Souce Code werden die Metriken ermittelt und ausgegeben. Eine (kompakte und auf
%das Wesentliche gekürzte) Ausgabe kann hier auf freiwilliger Basis aufgenommen
%werden und dient als Referenz bei Wartungsarbeiten (Degeneration des Codes und
%Designs).

%%%%%%%%%%%%%%%%%%%%%%%%%%%%%%%
\section{Quellcode}

\subsection{Object}
\cppmodul{CarPool/Object}

\subsection{CarPool}
\cppmodul{./CarPool/CarPool/CarPool}

\subsection{Vehicles}
\cppmodul{./CarPool/Vehicles/Vehicle}
\cppmodul{./CarPool/Vehicles/Car}
\cppmodul{./CarPool/Vehicles/Truck}
\cppmodul{./CarPool/Vehicles/Motorcycle}

\subsection{LogBook}
\cppmodul{./CarPool/LogBook/LogBook}

\subsection{main}
\cppfile{./CarPool/main}

% Um Quellcode einzufügen einfach diesen Befehl verwenden:
%\sourceCode{Relativer/Pfad/zum/SourceCode.Endung}

\end{document}